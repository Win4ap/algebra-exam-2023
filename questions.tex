\documentclass[12pt]{article}
\usepackage[utf8]{inputenc}
\usepackage[english,russian]{babel}
\usepackage{graphicx}
\usepackage{amsmath}
\usepackage{amsthm}
\usepackage{caption}
\usepackage{dsfont}
\usepackage{tikz}
\usepackage{amssymb}
\usepackage{subcaption}
\usepackage{imakeidx}
\usepackage{hyperref}
\usepackage[russian]{cleveref}
\usepackage[a4paper,left=15mm,right=15mm,top=30mm,bottom=20mm]{geometry}
\parindent=0mm
\parskip=3mm
\DeclareRobustCommand{\divby}{%
\mathrel{\text{\vbox{\baselineskip.65ex\lineskiplimit0pt\hbox{.}\hbox{.}\hbox{.}}}}%
}

\makeindex
\pagestyle{empty}
\title{Алгебра}
\author{ОПГ Допсята}
\date{\today}
\begin{document}
\maketitle
\large

Лекция №1.\\
\textbf{Свойства делимости в кольце целых чисел.}\\

\textbf{Теорема о делении с остатком.}\\

\textbf{Идеалы — определение, идеалы порожденные набором элементов,в целых числах все идеалы главные.}\\

\textbf{НОД и его единственность, теорема о его линейном представлении.}\\

\textbf{Критерий разрешимости линейного диофантова уравнения.}\\

\textbf{Алгоритм Евклида.}\\

Лекция №2.\\
\textbf{Взаимная простота (2 определения), лемма об отбрасывании взаимно простого множителя, простые числа и их основное свойство.}\\

\textbf{Основная теорема арифметики (существование и единственность). Контрпримеры к ОТА.}\\

\textbf{Степень вхождения и её свойства.}\\

\textbf{Делимость, свойство быть степенью в терминах канонического разложения, НОД и НОК.}\\

\textbf{Формулы для количества и суммы делителей.}\\

\textbf{Понятие группы, примеры абелевых групп.}\\

Лекция №3.\\
\textbf{Отношения, отображения, инъекция-сюръекция-биекция.}\\

\textbf{Основной пример группы: симметрическая группа. Кольцо, поле и близкие понятия.}\\

\textbf{Сравнимость по модулю сравнимость — отношение эквивалентности, согласованность с кольцевыми операциями, построение кольца вычетов.}\\

\textbf{Редукция по простому модулю, пример.}\\

\textbf{Обратимые элементы в кольцах вычетов, когда кольцо вычетов - поле.}\\
\textbf{Деление в кольцах вычетов примеры.}\\

Лекция №4.\\
\textbf{Сокращение в группах и кольцах, кольца без делителей нуля.}\\

\textbf{Решение Линейного диофантова уравнения.}\\

\textbf{Мультипликативная группа кольца. Порядок элемента, его свойства.}\\

\textbf{Теорема Лагранжа.}\\

\textbf{Малая теорема Ферма.}\\

\textbf{Циклическая группа, критерий цикличности. Подгруппа, подкольцо и т.п.}\\

\textbf{Гомоморфизмы групп и колец.}\\

Лекция №5.\\
\textbf{Изоморфизм групп, примеры.}\\

\textbf{Конечная циклическая группа изоморфна группе вычетов.}\\

\textbf{Группы простых порядков.}\\

\textbf{Прямое произведение групп и колец (и полей).}\\

\textbf{Китайская теорема об остатках (кольцевая версия).}\\

\textbf{Переформулировка в терминах систем сравнений, алгоритм решения системы.}\\

\textbf{Порядок мультипликативной группы: функция Эйлера, явная формула.}\\

Лекция №6.\\
\textbf{Мультипликативность функции Эйлера, другие мультипликативные функции. Теорема Эйлера, её улучшаемость.}\\

\textbf{Теорема о цикличности мультипликативных групп кольца вычетов, часть доказательства (сведения к случаю степени простого).}\\

\textbf{Первообразный корень по модулю $p^2$.}\\

\textbf{Биективность возведения в степень в кольце вычетов.}\\

\textbf{Алгоритм RSA.}\\

Лекция №7.\\
\textbf{Бесконечность простых и их частота.}\\

\textbf{Теорема Люка и тест Люка как хороший вероятностный тест.}\\

\textbf{Взламываемость простых чисел, полученных методом Люка.}\\

\textbf{Тест Ферма, абсолютно псевдопростые числа.}\\

\textbf{Тест Рабина Миллера и почему он вероятностно отсекает составные числа.}\\

\textbf{Первообразный корень по модулю $p^k$.}\\

\end{document}
